\documentclass{article}

\usepackage[T1]{fontenc}
\usepackage[utf8]{inputenc}    % change to your encoding (e.g. utf8)
\usepackage[ngerman]{babel}
\usepackage{lmodern}
\usepackage{microtype}
\usepackage{dsfont}
\usepackage{amsmath}


\usepackage{hyperref}


\title{Syntax und Semantik -- ein Tutorial}
\author{Jan S\"urmeli\\\url{suermeli@googlemail.com}}


\begin{document}
 
 \maketitle
 
 
 \section{Syntax vs. Semantik: Ideen haben und hinschreiben}

 In diesem Abschnitt untersuchen wir den Unterschied zwischen Syntax und 
Semantik. Kurz gesagt ist Syntax die \emph{Art}, Dinge hinzuschreiben, und 
Semantik die \emph{Bedeutung} des Geschriebenen. Den Unterschied untersuchen 
wir zunächst anhand einiger konkreter Beispiele. Anschließend formalisieren wir 
die Begriffe der Syntax und der Semantik auf eine recht abstrakte Weise. 

\paragraph{Erstes Beispiel: Programmiersprachen.} Bevor wir ein Programm 
schreiben, müssen wir uns für eine \emph{Programmiersprache} 
entscheiden. Damit wir ein Programm in einer Programmiersprache $L$ 
hinschreiben können, müssen wir die \emph{Syntax} von $L$ kennen: Ein Compiler 
(oder Interpreter) für die Sprache $L$ akzeptiert nur solche Quelltexte, die 
der Syntax von $L$ folgen. Die Syntax von Programmiersprachen umfasst unter 
anderem die Schlüsselwörter und die Namen eingebauter Datentypen. Ein Programm 
in $L$ ist jedoch (meist) mehr als eine bloße Aneinandereihung von 
Schlüsselwörtern. Tatsächlich ist festgelegt, welche \emph{Zeichenketten} in 
welcher Reihenfolge und Kombination stehen dürfen. Die 
Syntax einer Programmiersprache $L$ beantwortet also die Frage, \emph{ob ein 
gegebener Quelltext $Q$ ein Programm in $L$ ist}. Haben wir ein Programm $Q$ in 
$L$ geschrieben, möchten wir dieses meist ausführen. Dazu geben wir $Q$ in 
einen Compiler oder Interpreter, um $Q$ \emph{auszuführen}. Die Semantik von $L$ 
legt fest, was passiert, wenn wir $Q$ ausführen. Dazu legt die Semantik unter 
anderem fest, welche Schlüsselwörter welche Bedeutung haben. Genauer legt sie 
genau fest, \emph{welche Bedeutung $Q$ bezüglich der Programmiersprache 
$L$ hat}. Ein Quelltext $Q$ kann dabei ein Programm gleich zweier 
Programmiersprachen $L_1$ und $L_2$ sein, und in $L_1$ etwas völlig anderes 
bedeuten als in $L_2$. Ein gutes Beispiel dafür ist das 
Gleichheitszeichen, das in manchen Programmiersprachen \emph{Gleichheit} und in 
anderen \emph{Zuweisung} bedeutet. Dieser Punkt ist äußerst wichtig: Selbst 
wenn wir die Syntax einer Programmiersprache kennen, kennen wir nicht 
automatisch auch ihre Semantik. Auch wenn wir Syntax und Semantik einer 
Programmiersprache häufig gleichzeitig lernen, sind es tatsächlich zwei 
unterschiedliche Gedanken. 

\paragraph{Zweites Beispiel: Mathematische Funktionen}
In der Schule haben wir gelernt, dass wir eine Funktion $f$ hinschreiben 
können, in dem wir erst 
\begin{equation}
 f : \mathds{R}\rightarrow\mathds{R}
 \label{eq:ex-math-funcs-f-1}
\end{equation}
und dann 
\begin{equation}
 f(x) = x^2 + 6x + 2
 \label{eq:ex-math-funcs-f-2}
\end{equation}
schreiben. Wir haben auch gelernt, dass wir damit eine Funktion $f$ 
hinschreiben, die eine Parabel im zweidimensionalen Koordinatensystem 
beschreibt, oder etwas abstrakter: Eine Menge $P_f$ von Punkten im 
zweidimensionalen Koordinatensystem, wobei für jede reelle Zahl $x$ genau ein 
Punkt in $P_f$ existiert, dessen $X$-Koordinate $x$ ist. Außerdem ist 
die $Y$-Koordinate dieses einen Punktes genau $f(x)$, also das Ergebnis davon, 
$x$ in $x^2+6x+2$ einzusetzen. Angenommen, wir schreiben noch folgendes dazu: 
\begin{equation}
 g : \mathds{R}\rightarrow\mathds{R}
 \label{eq:ex-math-funcs-g-1}
\end{equation}
\begin{equation}
 g(x) = (x + 3)^2 - 7
 \label{eq:ex-math-funcs-g-2}
\end{equation}
Damit haben wir eine Funktion $g$ beschrieben -- eine Menge $P_g$ von Punkten 
im zweidimensionalen Koordinatensystem. Gucken 
wir ein bisschen länger auf die $P_f$ und $P_g$ stellen wir fest, dass sie 
\emph{gleich} sind: Für jede reelle Zahl $x$ hat der Punkt in $P_f$ mit 
$X$-Koordinate $x$ die gleiche $Y$-Koordinate wie der Punkt in $P_g$ mit 
$X$-Koordinate $x$. Was ist hier passiert? Wir haben die Punktmenge $P_f$ 
einmal durch die Zeichenketten \eqref{eq:ex-math-funcs-f-1} und 
\eqref{eq:ex-math-funcs-f-2} beschrieben. Anschließend haben wir dieselbe Menge 
von Punkten durch die Zeichenketten \eqref{eq:ex-math-funcs-g-1} und 
\eqref{eq:ex-math-funcs-g-2} beschrieben. Ähnlich zum vorherigen Beispiel 
haben wir eine Syntax gelernt, um Funktionen zu beschreiben, und gelernt, was 
eigentlich die Bedeutung solch einer Beschreibung ist. Diesmal haben wir den 
Fall kennen gelernt, dass zwei unterschiedliche Beschreibungen dieselbe 
Bedeutung haben können. 

\paragraph{Eine abstrakte Definition für Syntax und Semantik}
Eine \emph{Syntax} legt stets ein Universum erlaubter \emph{Zeichenketten} fest 
-- also nichts anderes als eine formale Sprache über einem Alphabet. 
Eine \emph{Semantik} weist jeder syntaktisch erlaubten Zeichenkette eine 
\emph{Bedeutung} zu. Eine Zeichenkette kann zu mehr als nur einer Syntax 
passen. Zwei Zeichenketten können bezüglich 
derselben Semantik dieselbe Bedeutung haben.

\section{Signaturen und Strukturen}

In diesem Abschnitt definieren wir konkret die Begriffe der Syntax und Semantik 
in der Welt der Algebra: Wir legen Syntax durch Signaturen, Semantik 
durch Strukturen fest. Dazu wenden wir uns zunächst der Semantik -- also den 
Strukturen -- zu. Anschließend betrachten wir die Syntax -- also die 
Signaturen. 

\subsection{Strukturen}

Wenn wir an die Schulmathematik denken, dreht sie sich doch größtenteils um 
Funktionen auf reellen Zahlen. Manchmal betrachten wir bestimmte Teilmengen der 
reellen Zahlen, wie die positiven reellen Zahlen oder die rationalen Zahlen. 
Manche dieser Funktion berechnen zu einer reellen Zahl eine andere reelle Zahl, 
wie zum Beispiel das Quadrieren. Andere Funktionen berechnen 
zu zwei reellen Zahlen eine reelle Zahl wie zum Beispiel die Addition. 

Wir haben jedoch auch andere Funktionen kennen gelernt, zum Beispiel die 
Ableitung einer Funktion: Die Ableitung einer Funktion $f$ über den reellen 
Zahlen ist die Funktion $f^\prime$, die jedem $x\in\mathds{R}$ die Steigung der 
Funktion $f$ im Punkt $x$ zuordnet. 

Was macht eine Funktion aus? Diese Frage wird in unterschiedlichen Werken 
unterschiedlich beantwortet. In diesem Tutorial wollen wir uns auf 
folgendes einigen: 
\begin{enumerate}
 \item Eine Funktion hat eine Stelligkeit $n$, wobei $n\in\{0,1,2,\ldots\}$.
 \item Eine $n$-stellige Funktion hat einen Definitionsbereich 
$A_1\times\ldots\times A_n$, wobei $A_1\ldots A_n$ Mengen sind. 
 \item Eine Funktion hat einen Wertebereich $A$. 
 \item Eine $0$-stellige Funktion $f$ heißt \emph{Konstante} und beschreibt 
einen Wert aus $A$. 
 \item Eine $n$-stellige Funktion $f$ mit $n \geq 1$ weist jedem Element aus 
dem Definitionsbereich $A_1\times\ldots\times A_n$ ein Element aus dem 
Wertebereich $A$ zu.
 \end{enumerate}

Jetzt sind wir bereit den Begriff der \emph{Struktur} einzuführen. Eine 
Struktur $S$ besteht aus endlich vielen Mengen $B_1\ldots B_\ell$ sowie 
endlichen vielen Funktionen $f_1\ldots f_k$, so dass für jedes $i$ mit $1\leq 
i\leq k$ gilt: $f_i$ ist eine $n$-stellige Funktion mit Definitionsbereich 
$A_1\times\ldots\times A_n$ und Wertebereich $A$, so dass
\begin{enumerate}
 \item $\{A_1,\ldots,A_n\}\subseteq\{B_1\ldots B_\ell\}$ und 
 \item $A\in\{B_1,\ldots,B_n\}$.  
\end{enumerate}

Wir benutzen für Strukturen gerne eine Tupel-Schreibweise: Sind 
$B_1,\ldots,B_n$ und $f_1,\ldots,f_k$ bekannt, schreiben wir $S$ auch als 
$(B_1,\ldots,B_n;f_1,\ldots,f_k)$. 

\subsection{Signaturen}

Als ein einführendes Beispiel betrachten wir drei Strukturen: 
\begin{enumerate}
 \item $S_1 = (\mathds{N};\mathrm{suc},0)$, wobei $\mathds{N}$ die Menge der 
natürlichen Zahlen und $\mathrm{suc}$ die einstellige Funktion ist, die jeder 
natürlichen Zahl ihren Nachfolger zuordnet. 
 \item $S_2 = (\mathds{R};\mathrm{abs}_{\mathds{R}},17)$, wobei $\mathds{R}$ 
die Menge der reellen Zahlen und $\mathrm{abs}_{\mathds{R}}$ die einstellige 
Funktion ist, die jeder reellen Zahl ihren Betrag zuordnet. 
  \item $S_3 = (\mathds{F}; \mathrm{diff}, \mathrm{quadriere})$, wobei 
$\mathds{F}$ die Menge der $1$-stelligen Polynom-Funktionen\footnote{Eine 
$1$-stellige Polynom-Funktion ist eine Funktion, die durch ein Polynom 
beschrieben wird.} über $\mathds{R}$, $\mathds{R}$ 
die Menge der reellen Zahlen, $\mathrm{diff}$ die Funktion ist, die jeder 
Funktion aus $\mathds{F}$ seine Ableitung zuordnet, und $\mathrm{quadriere}$ 
jeder Zahl ihr Quadrat zuordnet. 
\end{enumerate}
Wir beobachten, dass die drei Strukturen $S_1$, $S_2$ und $S_3$ folgende 
Gemeinsamkeit aufweisen: Jede der drei Strukturen besteht aus einer Menge 
zusammen mit einer einstelligen Funktion und einer Konstante. Formal sagen wir, 
dass $S_1$, $S_2$ und $S_3$ dieselbe \emph{Signatur} haben. Bevor wir das 
konkretisieren, 
betrachten wir 
ein etwas komplizierteres Beispiel: 
\begin{enumerate}
 \item $S_4 = (\mathds{Q},\mathds{Z};\mathrm{rundeAuf})$, wobei $\mathds{Q}$ 
die Menge der rationalen Zahlen, $\mathds{Z}$ die Menge der ganzen Zahlen und 
$\mathrm{rundeAuf}$ die Funktion ist, die jede rationale Zahl $x$ auf die 
kleinste ganze Zahl $y$ aufrundet, die nicht echt kleiner als $x$ ist. 
 \item $S_5 = (\mathds{Z},\mathds{N};\mathrm{abs}_{\mathds{Z}})$, wobei 
$\mathds{Z}$ die Menge der ganzen Zahlen, $\mathds{N}$ die Menge der 
natürlichen Zahlen $0,1,2,\ldots$ und $\mathrm{abs}_{\mathds{Z}}$ die Funktion 
ist, die jeder ganzen Zahl ihren Betrag zuordnet. 
\end{enumerate}
Hier beobachten wir, dass $S_4$ und $S_5$ jeweils aus zwei Mengen bestehen und 
einer einstelligen Funktion bestehen, wobei der Definitionsbereich die erste 
und der Wertebereich die zweite Menge ist. Auch hier sagen wir, dass $S_4$ und 
$S_5$ dieselbe \emph{Signatur} haben. 

Was ist eine Signatur? Wie bei den Funktionen finden wir hier unterschiedliche 
Definitionen in unterschiedlichen Büchern. Wir einigen uns auf die folgenden 
Axiome: 
\begin{enumerate}
 \item Eine Signatur $\Sigma$ besteht aus endlich vielen \emph{Symbolen}. 
 \item Jedes Symbol aus $\Sigma$ ist entweder ein \emph{Sortensymbol} oder ein 
\emph{Funktionssymbol}.
 \item Jedes Funktionssymbol $f$ aus $\Sigma$ hat eine Stelligkeit 
$\mathrm{ar}_\Sigma(f)\in\{0,1,2,\ldots\}$. 
 \item Jedes Funktionssymbol aus $\Sigma$ hat einen \emph{Typ} 
$\mathrm{typ}_\Sigma(f) = s_1\ldots s_n s$, wobei 
$n = \mathrm{ar}_\Sigma(f)$ und $s_1,\ldots,s_n,s$ jeweils Sortensymbole sind. 
\end{enumerate}

Wie Strukturen schreiben wir Signaturen gerne als Tupel, wobei wir zunächst 
alle Sortensymbole auflisten und nach einem Semikolon alle Funktionssymbole mit 
ihrem Typ auflisten. Am Typ erkennen wir auch leicht die Stelligkeit. Wir 
stellen im Folgenden den Zusammenhang zwischen Signaturen und Strukturen dar. 

Wie schon angedeutet, hat eine Struktur eine Signatur. Tatsächlich hat eine 
Struktur sogar unendlich viele Signaturen: Sei $\Sigma = 
(s_1,\ldots,s_k;f_1:\mathrm{typ}_\Sigma(f_1),\ldots,
f_\ell:\mathrm{typ}_\Sigma(f_\ell))$ eine Signatur
und $S = 
(B_1,\ldots,B_m;g_1,\ldots,g_n)$ eine Struktur. Wenn $k = m$ und $\ell = n$, 
dann bezeichne $S(s_i)$ die Menge $A_i$ für alle $1\leq i\leq m$. Dann ist $S$ 
eine \emph{$\Sigma$-Struktur}, falls: 
\begin{enumerate}
 \item $k = m$, 
 \item $\ell = n$,
 \item Für alle $1\leq i\leq n$ gilt: Sei $r = \mathrm{ar}_\Sigma(f_i)$ und  
$\mathrm{typ}_\Sigma(f_i) = t_1\ldots t_r t$. Dann ist $g$ $r$-stellig, 
$S(t_1)\times\ldots\times S(t_r)$ der Definitionsbereich von $g_i$ und $S(t)$ 
der Wertebereich von $g_i$. 
\end{enumerate}
Die Strukturen $S_1$, $S_2$ und $S_3$ sind jeweils $\Sigma_A$-Strukturen für 
$\Sigma_A =  (A;f:AA,c:A)$. Die Strukturen $S_4$ und $S_5$ sind jeweils 
$\Sigma_B$-Strukturen für $\Sigma_B = (A,B;f:AB)$. 












 
\end{document}
